\documentclass[11pt]{article}
    \title{\textbf{Homework 4}}
    \author{Andrei Tumbar}
    \date{02-24-2021}

\usepackage{circuitikzgit}
\usepackage{tikz}
\usepackage{multirow}
\usepackage{titlesec}
\usepackage{float}
\usepackage{pgfplots, pgfplotstable}
\usepackage{lmodern}
\usepackage{siunitx}
\usepackage{listings}
\usepackage[left=1in, right=1in, bottom=0in]{geometry}

%XeLaTeX packages
\usepackage{xltxtra}
\usepackage{fontspec} %Font package
\usepackage{xunicode}
\usepackage{inconsolata}

\setmonofont{inconsolata}

\addtolength{\topmargin}{-.875in}

\usepackage{amsmath}

\DeclareFontFamily{U}{mathx}{\hyphenchar\font45}
\DeclareFontShape{U}{mathx}{m}{n}{ <-> mathx10 }{}
\DeclareSymbolFont{mathx}{U}{mathx}{m}{n}
\DeclareFontSubstitution{U}{mathx}{m}{n}
\DeclareMathAccent{\widebar}{\mathalpha}{mathx}{"73}

\makeatletter
\newcommand{\cwidebar}[2][0]{{\mathpalette\@cwidebar{{#1}{#2}}}}
\newcommand{\@cwidebar}[2]{\@cwideb@r{#1}#2}
\newcommand{\@cwideb@r}[3]{%
  \sbox\z@{$\m@th#1\mkern-#2mu#3\mkern#2mu$}%
  \widebar{\box\z@}%
}
\makeatother

\usetikzlibrary{circuits.logic.IEC,calc}

\setlength{\parindent}{0pt}
\newcommand*\xor{\oplus}
\def\code#1{\texttt{#1}}
\lstset{basicstyle=\footnotesize\ttfamily,breaklines=true}


\ctikzset{tripoles/pmos style/emptycircle}

\titleformat{\section}
  {\normalfont\fontsize{12}{15}\bfseries}{\thesection}{1em}{}

\begin{document}

\maketitle
\thispagestyle{empty}

\section*{6.10}

Convert the following instructions into machine language.

\begin{lstlisting}
add $t0, $s0, $s1
lw $t0, 0x20($t7)
addi $s0, $0, -10
\end{lstlisting}

\begin{lstlisting}
000000|10000|10001|01000|00000|100000
100011|01111|01000|0000000000100000
001000|00000|10000|1111111111110110
\end{lstlisting}

\section*{6.12}
\subsection*{Which instructions from Exercise 6.10 are I-type instructions?}

The \code{lw} and \code{addi} instructions are I-type instructions.

\subsection*{Sign-extend the 16-bit immediate of each instruction from part (a) so that it becomes a 32-bit number.}
\begin{lstlisting}
(16-bit)         (32-bit)
0x0020       -> 0x00000020
-10 (0xff06) -> 0xffffff06
\end{lstlisting}

\subsection*{6.14}
\begin{lstlisting}[language=C]
(hex) 0x20080000
(bin) 001000|00000|01000|0000000000000000
(MIPS) addi $t0, $0, 0x0000

(hex) 0x20090001 
(bin) 001000|00000|01001|0000000000000001
(MIPS) addi $t1, $0, 0x0001

(hex) 0x0089502A
(bin) 000000|00100|01001|01010|00000|101010
(MIPS) slt $t2, $a0, $t1

(hex) 0x15400003
(bin) 000101|01010|00000|0000000000000011
(MIPS) bne $t2, $0, 0x0003

(hex) 0x01094020
(bin) 000000|01000|01001|01000|00000|100000
(MIPS) add $t0, $t0, $t1

(hex) 0x21290002
(bin) 001000|01001|01001|0000000000000010
(MIPS) addi $t1, $t1, 0x0002

(hex) 0x08100002
(bin) 000010|00,0001,0000,0000,0000,0000,0010
(addr) 0000|0000|0100|0000|0000|0000|0000|1000
(MIPS) j 0x0100002 (0x00400008)

(hex) 0x01001020
(bin) 000000|01000|00000|00010|00000|100000
(MIPS) add $v0, $t0, $0

(hex) 0x03E00008
(bin) 000000|11111|00000|00000|00000|001000
(MIPS) jr $ra


int f(int n)
{
    int t0 = 0;

    for (int t1 = 1; t1 < n; t1 += 2)
    {
        t0 += t1;
    }

    return t0;
}

\end{lstlisting}

This program will sum up all of the odd numbers that are less
than the input \code{n} and return this value.

\section*{Question 6.1}
Write assembly code to swap two registers.

\begin{lstlisting}
; Swap registers $t0 and $t1
xor $t0, $t0, $t1
xor $t1, $t1, $t0
xor $t0, $t0, $t1
\end{lstlisting}

\section*{Final Question}
Show all relevant data and control values on Harris figure 7.11 for commands
\begin{lstlisting}
add  $t0, $s0, $s1
addi $s0, $0, -10
\end{lstlisting}

\begin{table}[H]
\renewcommand{\arraystretch}{1.2}
\setlength{\tabcolsep}{12pt}
\caption{Relevant data values}
\begin{center}
\begin{tabular}{|c||c|c|c|c|}\hline
Instruction & \code{A1} & \code{A2} & \code{A3} & \code{SignImm} \\\hline
\code{add \$t0, \$s0, \$s1} & \code{10000} & \code{10001} & \code{01000} & N/A\\\hline
\code{addi \$s0, \$0, -10} & \code{00000} & N/A & \code{01000} & \code{0xfffffff6}\\\hline
\end{tabular}
\end{center}
\label{tab:TRUTH}
\end{table}

\begin{table}[H]
\renewcommand{\arraystretch}{1.2}
\setlength{\tabcolsep}{12pt}
\caption{Relevant control values}
\begin{center}
\begin{tabular}{|c||c|c|c|c|c|c|}\hline
Instruction & \code{MemToReg} & \code{MemWrite} & \code{Branch} & \code{ALUSrc} & \code{RegDst} & \code{RegWrite} \\\hline
\code{add \$t0, \$s0, \$s1} & \code{0} & \code{0} & \code{0} & \code{0} & \code{0} & \code{1} \\\hline
\code{addi \$s0, \$0, -10} & \code{0} & \code{0} & \code{0} & \code{1} & \code{0} & \code{1} \\\hline
\end{tabular}
\end{center}
\label{tab:TRUTH}
\end{table}

\end{document}

