\documentclass[11pt]{article}
    \title{\textbf{Homework 2}}
    \author{Andrei Tumbar}
    \date{}
    
    \addtolength{\topmargin}{-3cm}
    \addtolength{\textheight}{3cm}

\usepackage{circuitikzgit}
\usepackage{tikz}
\usepackage{multirow}
\usepackage{titlesec}
\usepackage{float}
\usepackage{pgfplots, pgfplotstable}
\usepackage{lmodern}
\usepackage{siunitx}

\usepackage{amsmath}

\DeclareFontFamily{U}{mathx}{\hyphenchar\font45}
\DeclareFontShape{U}{mathx}{m}{n}{ <-> mathx10 }{}
\DeclareSymbolFont{mathx}{U}{mathx}{m}{n}
\DeclareFontSubstitution{U}{mathx}{m}{n}
\DeclareMathAccent{\widebar}{\mathalpha}{mathx}{"73}

\makeatletter
\newcommand{\cwidebar}[2][0]{{\mathpalette\@cwidebar{{#1}{#2}}}}
\newcommand{\@cwidebar}[2]{\@cwideb@r{#1}#2}
\newcommand{\@cwideb@r}[3]{%
  \sbox\z@{$\m@th#1\mkern-#2mu#3\mkern#2mu$}%
  \widebar{\box\z@}%
}
\makeatother

\usetikzlibrary{circuits.logic.IEC,calc}

\setlength{\parindent}{0pt}
\newcommand*\xor{\oplus}

\begin{document}

\maketitle
\thispagestyle{empty}

\section*{Question 1}
\begin{equation}
f = (\widebar{x_1}\widebar{x_2}x_3) + (\widebar{x_1}x_2\widebar{x_3}) + (x_1\widebar{x_2}\widebar{x_3}) + (x_1x_2x_3)
\end{equation}

\begin{table}[h]
\renewcommand{\arraystretch}{1.2}
\setlength{\tabcolsep}{12pt}
\caption{Truth table for equation 1}
\begin{center}
\begin{tabular}{|c|c|c||c|}\hline
$x_1$ & $x_2$ & $x_3$ & Value\\\hline
0 & 0 & 0 & 0 \\\hline
0 & 0 & 1 & 1 \\\hline
0 & 1 & 0 & 1 \\\hline
0 & 1 & 1 & 0 \\\hline
1 & 0 & 0 & 1 \\\hline
1 & 0 & 1 & 0 \\\hline
1 & 1 & 0 & 0 \\\hline
1 & 1 & 1 & 1 \\\hline
\end{tabular}
\end{center}
\label{tab:TRUTH}
\end{table}

If this circuit was implemented as a CMOS gate, it would take $46$ transistors.
Each triple AND/OR gates take 8 transistors. Each invertor takes 2 transistors.
$8*5 + 2*3 = 46$ 
 
\section*{Question 3}
\begin{equation}
f = x_1 \xor x_2 \xor x_3
\end{equation}
This is logically equivalent to the truth table in question 1. The output $f$ is 
true when an odd number of inputs are HIGH. This is the same as a 3-input XOR gate.

\pagebreak

\ctikzset{logic ports=ieee}

\section*{Question 5}
\begin{figure}[!ht]
        \centering
        %! suppress = Ellipsis
        \begin{circuitikz}[circuit logic IEC]
        
        \foreach \i in {1,...,4} {
            \draw (0, 2*\i) node (nor\i) [ieeestd nor port] {};
        }
        
        \draw (3, 7) node(nand1) [nand port] {};
        \draw (nor4.out) -- ++(0.3, 0) |- (nand1.in 1);
        \draw (nor3.out) -- ++(0.3, 0) |- (nand1.in 2);

        \draw (3, 3) node(nand2) [nand port] {};
        \draw (nor2.out) -- ++(0.3, 0) |- (nand2.in 1);
        \draw (nor1.out) -- ++(0.3, 0) |- (nand2.in 2);

        \draw (6, 5) node(nand3) [nand port] {};
        \draw (nand1.out) -- ++(0.3, 0) |- (nand3.in 1);
        \draw (nand2.out) -- ++(0.3, 0) |- (nand3.in 2);
        
        \end{circuitikz}
\end{figure}


Each 2 input NOR/NAND gate takes 4 transistors. There are 4 NOR gates and 3 
NAND gates meaning there are $4*7$ or $28$ transistors total.

\end{document}

