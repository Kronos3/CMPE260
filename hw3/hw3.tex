\documentclass[11pt]{article}
    \title{\textbf{Homework 3}}
    \author{Andrei Tumbar}
    \date{}
    
    \addtolength{\topmargin}{-3cm}
    \addtolength{\textheight}{3cm}

\usepackage{circuitikzgit}
\usepackage{tikz}
\usepackage{multirow}
\usepackage{titlesec}
\usepackage{float}
\usepackage{pgfplots, pgfplotstable}
\usepackage{lmodern}
\usepackage{siunitx}

\usepackage{amsmath}

\DeclareFontFamily{U}{mathx}{\hyphenchar\font45}
\DeclareFontShape{U}{mathx}{m}{n}{ <-> mathx10 }{}
\DeclareSymbolFont{mathx}{U}{mathx}{m}{n}
\DeclareFontSubstitution{U}{mathx}{m}{n}
\DeclareMathAccent{\widebar}{\mathalpha}{mathx}{"73}

\makeatletter
\newcommand{\cwidebar}[2][0]{{\mathpalette\@cwidebar{{#1}{#2}}}}
\newcommand{\@cwidebar}[2]{\@cwideb@r{#1}#2}
\newcommand{\@cwideb@r}[3]{%
  \sbox\z@{$\m@th#1\mkern-#2mu#3\mkern#2mu$}%
  \widebar{\box\z@}%
}
\makeatother

\usetikzlibrary{circuits.logic.IEC,calc}

\setlength{\parindent}{0pt}
\newcommand*\xor{\oplus}

\ctikzset{tripoles/pmos style/emptycircle}

\begin{document}

\maketitle
\thispagestyle{empty}

\section*{84}
\subsubsection*{a)}

\begin{figure}[!ht]
    \centering
    %! suppress = Ellipsis
    \begin{circuitikz}[circuit logic IEC]
    
    \draw (5, 1) coordinate(Vsource) node[vcc]{VCC};
    \draw (5, -1) coordinate(OutConn);
    \draw (OutConn) -- (9, -1) node[label=right:Out]{};

    \foreach \i/\name in {1/A,2/B,3/C,4/D} {
        \draw (2*\i, 0) node (pmos\i) [pmos] {};
        \draw (5, -1.5*\i - 0.5) node (nmos\i) [nmos] {};

		% Connect source + drain for pmos
        \draw (OutConn) -| (pmos\i.S);
        \draw (Vsource) -| (pmos\i.D);
        
        \draw (pmos\i.G) node[label=left:\name]{};
        \draw (nmos\i.G) node[label=left:\name]{};
    }
    
    \draw (nmos1.D) -- (OutConn);
    \draw (nmos4.S) node[ground](GND){};
    
    \end{circuitikz}
    \caption{Four-input NAND gate}
\end{figure}


\pagebreak

\subsubsection*{b)}

\begin{figure}[!ht]
    \centering
    %! suppress = Ellipsis
    \begin{circuitikz}[circuit logic IEC]
    
    \draw (0, 0) node (pmosb)[pmos] {};
    \draw (2, 0) node (pmosc)[pmos] {};
    \draw (1, -2) node (pmosa)[pmos] {};
    
    \draw (pmosb.S) node[vcc]{VCC};
    \draw (pmosb.S) -- (pmosc.S);
    \draw (pmosb.D) -| (pmosa.S);
    \draw (pmosc.D) -| (pmosa.S);
    
    \draw (pmosa.G) node[label=left:C]{};
    \draw (pmosb.G) node[label=left:A]{};
    \draw (pmosc.G) node[label=left:B]{};

    \draw (0, -4) node (nmosb)[nmos] {};
    \draw (0, -6) node (nmosc)[nmos] {};
    \draw (2, -5) node (nmosa)[nmos] {};
    
    \draw (nmosa.G) node[label=left:C]{};
    \draw (nmosb.G) node[label=left:A]{};
    \draw (nmosc.G) node[label=left:B]{};
    
    \draw (nmosb.S) -- (nmosc.D);
    \draw (nmosb.D) |- (pmosa.D);
    \draw (nmosa.D) |- (pmosa.D);
    \draw (nmosa.S) |- (nmosc.S);
    
    \draw (pmosa.D) ++(1,0) -- ++(0.5,0) node[label=right:Out]{};
    \draw (nmosc.S) node[ground](GND){};
    
    \end{circuitikz}
    \caption{Three-input OR-AND-INVERT gate}
\end{figure} 

\pagebreak
\subsubsection*{c)}

\begin{figure}[H]
    \centering
    %! suppress = Ellipsis
    \begin{circuitikz}[circuit logic IEC]
    
    \draw (0, 0) node (nmos1)[nmos] {};
    \draw (0, 2) node (nmos2)[nmos] {};
    \draw (2, 1) node (nmos3)[nmos] {};
    \draw (nmos1.G) node[label=left:B]{};
    \draw (nmos2.G) node[label=left:A]{};
    \draw (nmos3.G) node[label=left:C]{};

    \draw (nmos2.D) node[vcc]{VCC};
    \draw (nmos1.S) ++(2,0) -- ++(0.5, 0) node[label=right:Out]{};
    
    \draw (nmos1.D) -- (nmos2.S);
    \draw (nmos2.D) -| (nmos3.D);
    \draw (nmos1.S) -| (nmos3.S);
    
    \draw (0, -2) node (pmos1)[pmos] {};
    \draw (2, -2) node (pmos2)[pmos] {};
    \draw (1, -4) node (pmos3)[pmos] {};
    \draw (pmos1.G) node[label=left:A]{};
    \draw (pmos2.G) node[label=left:B]{};
    \draw (pmos3.G) node[label=left:C]{};

    \draw (pmos1.S) -- (nmos1.S);
    \draw (pmos2.S) -- (nmos3.S);
    \draw (pmos1.D) -- (pmos2.D);
    \draw (pmos3.S) |- (pmos1.D);
    \draw (pmos3.D) node[ground](GND){};
    
    \end{circuitikz}
    \caption{Three input AND-OR gate}
\end{figure} 

\pagebreak
\section*{85}
Sketch a transistor-level circuit for the following CMOS gates. Use the minimum
number of transistors.

\subsubsection*{a)}
\begin{figure}[H]
    \centering
    \begin{circuitikz}[circuit logic IEC]
    
    \foreach \i/\name in {1/A,2/B,3/C} {
    	\draw (0, -1.5*\i) node (pmos\name) [pmos] {};
    	\draw (pmos\name.G) node[label=left:\name]{};
    	
    	\draw (-4 + 2*\i, -6.5) node(nmos\name) [nmos] {};
    	\draw (nmos\name.G) node[label=left:\name]{};
    }
    
    \draw (pmosA.S) node[vcc]{VCC};
    \draw (nmosA.D) -- (nmosC.D);
    \draw (nmosA.S) node[ground](GND){} -- (nmosC.S);
    \draw (nmosB.D) -- (pmosC.D);
    \draw (nmosC.D) -- ++(0.5,0) node[label=right:Out]{};
    
    \end{circuitikz}
    \caption{Three-input NOR gate.}
\end{figure} 

\begin{figure}[H]
    \centering
    \begin{circuitikz}[circuit logic IEC]
    
    \foreach \i/\name in {1/A,2/B,3/C} {
    	\draw (0, -1.5*\i) node (nmos\name) [nmos] {};
    	\draw (nmos\name.G) node[label=left:\name]{};
    	
    	\draw (-4 + 2*\i, -6.5) node(pmos\name) [pmos] {};
    	\draw (pmos\name.G) node[label=left:\name]{};
    }
    
    \draw (nmosA.D) node[vcc]{VCC};
    \draw (pmosA.S) -- (pmosC.S);
    \draw (pmosA.D) node[ground](GND){} -- (pmosC.D);
    \draw (pmosB.S) -- (nmosC.S);
    \draw (pmosC.S) -- ++(0.5,0) node[label=right:Out]{};
    
    \end{circuitikz}
    \caption{Three-input AND gate}
\end{figure} 

\begin{figure}[H]
    \centering
    \begin{circuitikz}[circuit logic IEC]
    
    \foreach \i/\name in {1/A,2/B} {
    	\draw (2*\i - 3, 0) node (nmos\name) [nmos] {};
    	\draw (nmos\name.G) node[label=left:\name]{};
    	
    	\draw (0, -.3-1.5*\i) node(pmos\name) [pmos] {};
    	\draw (pmos\name.G) node[label=left:\name]{};
    }
    
    \draw (nmosA.D) node[vcc]{VCC};
    \draw (pmosB.D) node[ground](GND){};
    \draw (nmosB.S) |- (pmosA.S);
    \draw (nmosA.S) |- (pmosA.S);
    \draw (nmosA.D) -- (nmosB.D);
    \draw (pmosA.S) -- ++(1.5,0) node[label=right:Out]{};
    
    \end{circuitikz}
    \caption{Two-input OR gate}
\end{figure} 

\pagebreak

\section*{87}
\begin{table}[H]
\renewcommand{\arraystretch}{1.2}
\setlength{\tabcolsep}{12pt}
\caption{Truth table for Figure 1.50}
\begin{center}
\begin{tabular}{|c|c||c|}\hline
A & B & Y\\\hline
0 & 0 & 0 \\\hline
0 & 1 & 1 \\\hline
1 & 0 & 1 \\\hline
1 & 1 & 0 \\\hline
\end{tabular}
\end{center}
\label{tab:t1}
\end{table}
This is an XOR gate.

\section*{88}
\begin{table}[H]
\renewcommand{\arraystretch}{1.2}
\setlength{\tabcolsep}{12pt}
\caption{Truth table for equation 1}
\begin{center}
\begin{tabular}{|c|c|c||c|}\hline
A & B & C & Y\\\hline
0 & 0 & 0 & 1 \\\hline
0 & 0 & 1 & 0 \\\hline
0 & 1 & 0 & 1 \\\hline
0 & 1 & 1 & 0 \\\hline
1 & 0 & 0 & 1 \\\hline
1 & 0 & 1 & 0 \\\hline
1 & 1 & 0 & 0 \\\hline
1 & 1 & 1 & 0 \\\hline
\end{tabular}
\end{center}
\label{tab:TRUTH}
\end{table}

\pagebreak

\section*{Interview Question}
Sketch a transistor-level circuit for a CMOS four-input NOR gate.

\begin{figure}[!ht]
    \centering
    %! suppress = Ellipsis
    \begin{circuitikz}[circuit logic IEC]
    
    \draw (5, -1) coordinate(Vsource) node[vcc]{VCC};

    \foreach \i/\name in {1/A,2/B,3/C,4/D} {
        \draw (5, -1.5*\i) node (pmos\name) [pmos] {};
        \draw (2*\i, -8) node (nmos\name) [nmos] {};
        
        \draw (pmos\name.G) node[label=left:\name]{};
        \draw (nmos\name.G) node[label=left:\name]{};
    }

    \draw (nmosD.S) node[ground](GND){};
    \draw (nmosA.D) -- (nmosD.D);
    \draw (nmosA.S) -- (nmosD.S);
    \draw (pmosD.D) |- (nmosB.D);
    \draw (nmosD.D) -- ++(0.5,0) node[label=right:Out]{};
    
    \end{circuitikz}
    \caption{Four-input NORq gate}
\end{figure}

\end{document}

