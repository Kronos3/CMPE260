
\documentclass{article}
    \title{\textbf{Homework 12}}
    \author{Andrei Tumbar}
    \date{04-28-2021}


\usepackage{circuitikzgit}
\usepackage{tikz}
\usepackage{multirow}
\usepackage{titlesec}
\usepackage{float}
\usepackage{pgfplots, pgfplotstable}
\usepackage{lmodern}
\usepackage{makecell}
\usepackage{siunitx}
\usepackage{listings}
\usepackage{subcaption}
\usepackage{kmap}
\usepackage[left=1in, right=1in, bottom=0in]{geometry}

%XeLaTeX packages
\usepackage{xltxtra}
\usepackage{fontspec} %Font package
\usepackage{xunicode}
\usepackage{inconsolata}

\setmonofont{inconsolata}

\addtolength{\topmargin}{-.875in}

\usepackage{amsmath}

\DeclareFontFamily{U}{mathx}{\hyphenchar\font45}
\DeclareFontShape{U}{mathx}{m}{n}{ <-> mathx10 }{}
\DeclareSymbolFont{mathx}{U}{mathx}{m}{n}
\DeclareFontSubstitution{U}{mathx}{m}{n}
\DeclareMathAccent{\widebar}{\mathalpha}{mathx}{"73}

\makeatletter
\newcommand{\cwidebar}[2][0]{{\mathpalette\@cwidebar{{#1}{#2}}}}
\newcommand{\@cwidebar}[2]{\@cwideb@r{#1}#2}
\newcommand{\@cwideb@r}[3]{%
  \sbox\z@{$\m@th#1\mkern-#2mu#3\mkern#2mu$}%
  \widebar{\box\z@}%
}
\newcommand\currentcoordinate{\the\tikz@lastxsaved,\the\tikz@lastysaved}
\makeatother

\usetikzlibrary{circuits.logic.IEC,calc}

\setlength{\parindent}{0pt}
\newcommand*\xor{\oplus}
\def\code#1{\texttt{#1}}
\lstset{basicstyle=\footnotesize\ttfamily,breaklines=true}

\ctikzset{logic ports=ieee}
\ctikzset{tripoles/pmos style/emptycircle}
\tikzset{dff/.style={flipflop, flipflop def={
    t1=D, t2=clk, t6=Q, nd=1}},
}
\ctikzset{
        logic ports/scale=0.7,
}

\titleformat{\section}
  {\normalfont\fontsize{12}{15}\bfseries}{\thesection}{1em}{}

\begin{document}

\maketitle
\thispagestyle{empty}

\tikzset{
  ROMdot/.pic={
    \begin{scope}
    \fill (0,0) circle [radius=1mm];
    \end{scope}
  }
}

\newcommand*{\xMin}{0}%
\newcommand*{\xMax}{10}%
\newcommand*{\yMinA}{-15}%
\newcommand*{\yMinB}{-7}%
\newcommand*{\yMax}{0}%
\newcommand*{\topLabels}{
	A,$\widebar{A}$,
	B,$\widebar{B}$,
	C,$\widebar{C}$,
	D,$\widebar{D}$,
	\textbf{X},\textbf{Y},\textbf{Z}}

\section*{Exercise 5.51}

\begin{center}
\begin{tikzpicture}
	
	\foreach \topname [count=\i from 0] in \topLabels {
		\draw [very thin,gray] (\i,\yMinA) -- (\i,\yMax+1);
		\draw node [above] at (\i,\yMax+1) {\i}
			node [above] at (\i,\yMax+2) {\topname};
	}
    \foreach \i[evaluate=\i as \xi using int(\yMax-\i)] in {\yMinA,...,\yMax} {
        \draw [very thin,gray] (\xMin,\i) -- (\xMax,\i);
        \draw node [left] at (\xMin-0.5,\i) {$\xi$};
    }

	\draw (7.5,2) -- (7.5,\yMinA-0.5);

	\draw
		(1,0) pic{ROMdot}
		(3,0) pic{ROMdot}
		(5,0) pic{ROMdot}
		(7,0) pic{ROMdot}
		(1,-1) pic{ROMdot}
		(3,-1) pic{ROMdot}
		(5,-1) pic{ROMdot}
		(6,-1) pic{ROMdot}
		(1,-2) pic{ROMdot}
		(3,-2) pic{ROMdot}
		(4,-2) pic{ROMdot}
		(7,-2) pic{ROMdot}
		(1,-3) pic{ROMdot}
		(3,-3) pic{ROMdot}
		(4,-3) pic{ROMdot}
		(6,-3) pic{ROMdot}
		(1,-4) pic{ROMdot}
		(2,-4) pic{ROMdot}
		(5,-4) pic{ROMdot}
		(7,-4) pic{ROMdot}
		(1,-5) pic{ROMdot}
		(2,-5) pic{ROMdot}
		(5,-5) pic{ROMdot}
		(6,-5) pic{ROMdot}
		(1,-6) pic{ROMdot}
		(2,-6) pic{ROMdot}
		(4,-6) pic{ROMdot}
		(7,-6) pic{ROMdot}
		(1,-7) pic{ROMdot}
		(2,-7) pic{ROMdot}
		(4,-7) pic{ROMdot}
		(6,-7) pic{ROMdot}
		(0,-8) pic{ROMdot}
		(3,-8) pic{ROMdot}
		(5,-8) pic{ROMdot}
		(7,-8) pic{ROMdot}
		(0,-9) pic{ROMdot}
		(3,-9) pic{ROMdot}
		(5,-9) pic{ROMdot}
		(6,-9) pic{ROMdot}
		(0,-10) pic{ROMdot}
		(3,-10) pic{ROMdot}
		(4,-10) pic{ROMdot}
		(7,-10) pic{ROMdot}
		(0,-11) pic{ROMdot}
		(3,-11) pic{ROMdot}
		(4,-11) pic{ROMdot}
		(6,-11) pic{ROMdot}
		(0,-12) pic{ROMdot}
		(2,-12) pic{ROMdot}
		(5,-12) pic{ROMdot}
		(7,-12) pic{ROMdot}
		(0,-13) pic{ROMdot}
		(2,-13) pic{ROMdot}
		(5,-13) pic{ROMdot}
		(6,-13) pic{ROMdot}
		(0,-14) pic{ROMdot}
		(2,-14) pic{ROMdot}
		(4,-14) pic{ROMdot}
		(7,-14) pic{ROMdot}
		(0,-15) pic{ROMdot}
		(2,-15) pic{ROMdot}
		(4,-15) pic{ROMdot}
		(6,-15) pic{ROMdot}
  	;
  	
  	% Or up the results of X
  \draw (8,-12) pic{ROMdot} % AB
  		(8,-13) pic{ROMdot} % AB
  		(8,-14) pic{ROMdot} % AB
  		(8,-15) pic{ROMdot} % AB
  		
  		% B(not C)D
  		(8,-5) pic{ROMdot} 
  		(8,-13) pic{ROMdot}
  		
  		% (not A)(not B)
  		(8,0) pic{ROMdot}
  		(8,-1) pic{ROMdot}
  		(8,-2) pic{ROMdot}
  		(8,-3) pic{ROMdot}
  ;
  	
  	% Or up the results of Y
  \draw % AB
  		(9,-12) pic{ROMdot}
  		(9,-13) pic{ROMdot}
  		(9,-14) pic{ROMdot}
  		(9,-15) pic{ROMdot}
  		
  		% BD
  		(9,-5) pic{ROMdot}
  		(9,-7) pic{ROMdot} 
  		(9,-13) pic{ROMdot}
  		(9,-15) pic{ROMdot}
  ;
  	
  	% Or up the results of Z
  \draw
  		(10,-1) pic{ROMdot}
  		(10,-2) pic{ROMdot} 
  		(10,-3) pic{ROMdot}
  		(10,-4) pic{ROMdot}
  		(10,-5) pic{ROMdot}
  		(10,-6) pic{ROMdot}
  		(10,-7) pic{ROMdot}
  		(10,-8) pic{ROMdot}
  		(10,-9) pic{ROMdot}
  		(10,-10) pic{ROMdot}
  		(10,-11) pic{ROMdot}
  		(10,-12) pic{ROMdot}
  		(10,-13) pic{ROMdot}
  		(10,-14) pic{ROMdot}
  		(10,-15) pic{ROMdot}
  ;

\end{tikzpicture}
\end{center}

\section*{Exercise 5.52}

\begin{center}
\begin{tikzpicture}
	
	\foreach \topname [count=\i from 0] in \topLabels {
		\draw [very thin,gray] (\i,\yMinB) -- (\i,\yMax+1);
		\draw node [above] at (\i,\yMax+1) {\i}
			node [above] at (\i,\yMax+2) {\topname};
	}
    \foreach \i[evaluate=\i as \xi using int(\yMax-\i)] in {\yMinB,...,\yMax} {
        \draw [very thin,gray] (\xMin,\i) -- (\xMax,\i);
        \draw node [left] at (\xMin-0.5,\i) {$\xi$};
    }

	\draw (7.5,2) -- (7.5,\yMinB-0.5);

	% AB
  \draw (0,0) pic{ROMdot} % A
  		(2,0) pic{ROMdot} % B
  ;
  
  % B(not C)D
  \draw (2,-1) pic{ROMdot} % B
  		(5,-1) pic{ROMdot} % not C
  		(6,-1) pic{ROMdot} % D
  ;
  
  % (not A)(not B)
  \draw (1,-2) pic{ROMdot} % not A
  		(3,-2) pic{ROMdot} % not B
  ;
  
  % BD
  \draw (2,-3) pic{ROMdot} % B
  		(6,-3) pic{ROMdot} % D
  ;
  
  \draw (0,-4) pic{ROMdot}; % A
  \draw (2,-5) pic{ROMdot}; % B
  \draw (4,-6) pic{ROMdot}; % C
  \draw (6,-7) pic{ROMdot}; % D

  % Or up the results of X
  \draw (8,0) pic{ROMdot}
  		(8,-1) pic{ROMdot}
  		(8,-2) pic{ROMdot}
  ;
  
  % Or up the results of Y
  \draw (9,-0) pic{ROMdot}
  		(9,-3) pic{ROMdot}
  ;
  
  % Or up the results of Z
  \draw (10,-4) pic{ROMdot}
  		(10,-5) pic{ROMdot}
  		(10,-6) pic{ROMdot}
  		(10,-7) pic{ROMdot}
  ;
\end{tikzpicture}
\end{center}

\section*{Exercise 5.53}

Size of ROM to implement function

\subsection*{16-bit adder/subtractor with $C_{in}$ and $C_{out}$}
Two 16-bit inputs and a carry-in: 33 inputs.
8589934592 x 17

\subsection*{8x8 multipier}
Two 8 bit inputs, 16-bit output.
65536 x 16

\subsection*{16-bit priority encoder}
Can encode 4 numbers.
65536 x 4

\section*{Explanation}
These are impractical to implement with a ROM because there
is LOTS of redundancy in the ROM grid. These grids are far
larger than they would be with more efficient reprogrammable
logic options.


\pagebreak
\section*{Exercise 5.55}

\begin{table}[H]
    \renewcommand{\arraystretch}{1.2}
    \caption{LUT4 logic elements required to implement the functions.}
    \begin{center}
        \begin{tabular}{|c|c|}
            \hline
            Function & LE\\\hline
            \hline
            Exercise 2.13(c) & 1\\\hline
            Exercise 2.17(c) & 1\\\hline
            the two-output function from Exercise 2.24 & 2\\\hline
            the function from Exercise 2.35 & 2\\\hline
            a four-input priority encoder (see Exercise 2.36) & 2\\\hline
        \end{tabular}
    \end{center}
    \label{tab:decode}
\end{table}

\tikzset{
  LUT/.pic={
    \begin{scope}
    \draw (-0.5,1) rectangle (0.5,-1);
    \end{scope}
  }
}

\begin{figure}[h!]
\begin{center}
\begin{tikzpicture}
	
	\draw (0,0) coordinate(A) pic{LUT}
		(0,0) ++(-0.5,0.75) node[left]{A}
		(0,0) ++(-0.5,0.25)node[left]{B}
		(0,0) ++(-0.5,-0.25)node[left]{C}
		(0,0) ++(-0.5,-0.75)node[left]{D}
		(0,0) ++(0.5,0)node[right]{Y};
	\draw (A) ++(0,1) node[above] {LUT for 1 and 2};
	
	\draw (3,0) pic{LUT} coordinate(B)
		(B) ++(-0.5,0.75) node[left]{A}
		(B) ++(-0.5,0.25)node[left]{B}
		(B) ++(-0.5,-0.25)node[left]{C}
		(B) ++(-0.5,-0.75)node[left]{D}
		(B) ++(0.5,0)node[right]{Y};
	\draw (B) ++(1,1) node[above] {LUTs for 3 and 4};
	
	\draw (5.5,0) pic{LUT} coordinate(C)
		(C) ++(-0.5,0.75) node[left]{A}
		(C) ++(-0.5,0.25)node[left]{B}
		(C) ++(-0.5,-0.25)node[left]{C}
		(C) ++(-0.5,-0.75)node[left]{D}
		(C) ++(0.5,0)node[right]{Z};
	
	\draw (0,-3) pic{LUT} coordinate(D1)
		(D1) ++(-0.5,0.75) node[left]{A}
		(D1) ++(-0.5,0.25)node[left]{B}
		(D1) ++(-0.5,-0.25)node[left]{C}
		(D1) ++(-0.5,-0.75)node[left]{D}
		(D1) ++(0.5,0)node[right]{$Y_0$};
	\draw (3,-3) pic{LUT} coordinate(D2)
		(D2) ++(-0.5,0.75) node[left]{A}
		(D2) ++(-0.5,0.25)node[left]{B}
		(D2) ++(-0.5,-0.25)node[left]{C}
		(D2) ++(-0.5,-0.75)node[left]{D}
		(D2) ++(0.5,0)node[right]{$Y_1$};
	\draw (D1) ++(2,1) node[above] {LUTs for priority encoder};

\end{tikzpicture}
\end{center}
\caption{LUT configurations from Table 1}
\end{figure}

\end{document}

