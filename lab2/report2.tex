% A skeleton file for producing Computer Engineering reports
% https://kgcoe-git.rit.edu/jgm6496/KGCOEReport_template

\documentclass[CMPE]{../KGCOEReport}

% The following should be changed to represent your personal information
\newcommand{\classCode}{CMPE 260}  % 4 char code with number
\newcommand{\name}{Andrei Tumbar}
\newcommand{\LabSectionNum}{1}
\newcommand{\LabInstructor}{Moskal}	% The slash is to tell LaTeX that the period is between words
% not sentences so it spaces correctly. It won't appear in the
% final pdf
\newcommand{\TAs}{Jacob Meyerson}
\newcommand{\LectureSectionNum}{2}
\newcommand{\LectureInstructor}{Cliver}
\newcommand{\exerciseNumber}{2}
\newcommand{\exerciseDescription}{Register File}
\newcommand{\dateDone}{February 14th}
\newcommand{\dateSubmitted}{February 23rd}

\usepackage{tikz}
\usepackage{circuitikz}
\usetikzlibrary{calc}
\usetikzlibrary{circuits.logic.IEC,calc}
\usepackage{multirow}
\usepackage{titlesec}
\usepackage{float}
\usepackage{lmodern}
\usepackage{siunitx}
\usepackage{subcaption}
\usepackage{graphicx}
\usepackage[usestackEOL]{stackengine}
\usepackage{scalerel}
\usepackage[T1]{fontenc}
\usepackage{amsmath}

\def\lbar#1{\ThisStyle{%
    \setbox0=\hbox{$\SavedStyle#1$}%
    \stackengine{2.2\LMpt}{$\SavedStyle#1$}{\rule{\wd0}{0.1\LMpt}}{O}{c}{F}{F}{S}%
}}

\DeclareFontFamily{U}{mathx}{\hyphenchar\font45}
\DeclareFontShape{U}{mathx}{m}{n}{ <-> mathx10 }{}
\DeclareSymbolFont{mathx}{U}{mathx}{m}{n}
\DeclareFontSubstitution{U}{mathx}{m}{n}
\DeclareMathAccent{\widebar}{\mathalpha}{mathx}{"73}

\makeatletter
\newcommand{\cwidebar}[2][0]{{\mathpalette\@cwidebar{{#1}{#2}}}}
\newcommand{\@cwidebar}[2]{\@cwideb@r{#1}#2}
\newcommand{\@cwideb@r}[3]{%
    \sbox\z@{$\m@th\mkern-#2mu#3\mkern#2mu$}%
    \widebar{\box\z@}%
}
\newcommand\currentcoordinate{\the\tikz@lastxsaved,\the\tikz@lastysaved}
\makeatother

\newcommand\decbin[9]{%
    \par\smallskip
    \makebox[3cm][r]{$#1$\ }\fbox{#2}\,\fbox{#3}\,\fbox{#4}\,\fbox{#5}\,\fbox{#6}\,\fbox{#7}\,\fbox{#8}\,\fbox{#9}\par}


\def\code#1{\texttt{#1}}

\begin{document}
    \maketitle
    \section*{Abstract}

    In this laboratory exercise, a parameterized register file was created.
    Generic parameters were used to control the number and size of registers
    included in the register file.
    A test-bench was written to assert the functionality of the register by
    creating a table of inputs and expected outputs.
    Expected outputs were checked against simulation outputs and comparison
    failures tripped an assertion failure.

    \section*{Design Methodology}
    
    A block diagram of a 3-bit \code{LOG\_PORT\_DEPTH} 8-bit \code{BIT\_DEPTH}
    register was created to simplify the design of the generic parameter
    register file.

    \begin{figure}[H]
        \centering
        \vspace{-0.1in}
        %! suppress = Ellipsis
        \begin{circuitikz}[american, circuit logic IEC]

            \tikzset{mux8/.style={muxdemux,muxdemux def={Lh=8, NL=8, Rh=4, NB=1, w=1, square pins=0}}}
            \tikzset{mux2/.style={muxdemux,muxdemux def={Lh=2, NL=2, Rh=2, NB=1, w=1, square pins=0}}}
            \tikzset{register/.style={qfpchip,
            hide numbers, num pins=8, external pins width=0}}

            \def\RegMax{7};
            \def\RegSpacing{1.8}
            \def\RegYStart{10}

            \def\RegPinLabels{
                1/wd/right,
                2/clk/right,
                6/rd/left,
                7/we/below};

            \foreach \i in {0,...,\RegMax} {
                \draw (5, -1 * \i * \RegSpacing + \RegYStart) node[register](R\i){R\i};

                \foreach \pin/\pinname/\labelpos in \RegPinLabels {
                    \node [\labelpos, font=\tiny] at (R\i.bpin \pin) {\pinname};
                    \draw (R\i.bpin \pin) coordinate(R\i_\pinname);
                }
            }

            % Draw the input nodes
            \draw (0,0) coordinate(addr1);
            \draw (0,-0.5) coordinate(addr2);
            \draw (0, 4) coordinate(wdin);
            \draw (1.4, 5.5) coordinate(we);
            \draw (we) ++(-2, 0) coordinate(wein);

            \draw (addr1) to[font=\tiny, multiwire, l_=\code{LOG\_PORT\_DEPTH}] ++(-0.5,0) node [anchor=east]{Addr1};
            \draw (addr2) to[multiwire] ++(-0.5,0) node [anchor=east]{Addr2};
            \draw (wdin) to[font=\tiny, multiwire, l_=\code{BIT\_DEPTH}] ++(-0.5,0) node [anchor=east]{wd};
            \draw (wein) node [anchor=east]{we};

            \node [mux8](muxRd1) at (9.5, 8) {};
            \node [mux8](muxRd2) at (9.5, 2) {};

            % Connect the rd of every register to the two RDout muxes
            \def\MuxOffsetA{0.1}
            \def\MuxOffsetB{0.4}
            \def\WireSpacing{0.15}
            \foreach \i [evaluate=\i as \pin using int(\i+1)] in {0,...,\RegMax} {
                \draw (R\i_rd) -- ++(\MuxOffsetA + \i*\WireSpacing, 0) |- (muxRd1.lpin \pin);
                \draw (muxRd1.lpin \pin) ++(-\MuxOffsetB - \WireSpacing*\i, 0)
                        to [short, *-] (\currentcoordinate |- muxRd2.lpin \pin) -- (muxRd2.lpin \pin);
            }

            % Connect Addr1 and Addr2 to the Mux selects
            \foreach \i in {1,...,2} {
                \draw (addr\i) -| ++(2 - \WireSpacing*\i, -3.5) coordinate(addr\i_inter);
            }
            \draw (addr2_inter) -| (muxRd2.bpin 1);
            \draw (addr1_inter) -- (\currentcoordinate -| muxRd2.bpin 1) -- ++(1, 0) |- (muxRd1.bpin 1);

            % Connect the wd input to every data line for registers R0-R7
            \draw (wdin) -- ++(3.8, 0) coordinate(wdconnect);
            \foreach \i in {0,...,\RegMax} {
                \draw (wdconnect) to [short, *-*] (\currentcoordinate |- R\i_wd) -- (R\i_wd);
            }

            % Draw a decoder and label the outputs
            \def\RegWeSpace{0.2}
            \def\AndWeOffset{0.2}

            \node[and gate,inputs={nnn},and gate IEC symbol={Decoder},text height=5cm,text width=2cm] (A) at (0, 9) {};
            \draw  ([xshift=-10pt]A.input 2) node[left] {Addr3} -- (A.input 2);
            \foreach \C/\B [count=\i from 0] in {0.111/000,.222/001,.333/010,.444/011,.555/100,.666/101,.777/110,.888/111} {
                \draw ( $ (A.north east)!\C!(A.south east) $ ) -- ++(10pt,0) node[left,xshift=-10] {$\B$} coordinate (decoder_\i);

                % Place a corresponding AND gate
                % Connect the inputs
                \draw (decoder_\i) node[and port,scale=0.4, anchor=in 1] (and_we_\i) {};
                \draw (and_we_\i.in 2) to [short, -*] (we |- \currentcoordinate)
                     coordinate(weconn_\i);

                % Connect the outputs
                \draw (and_we_\i.out) -- ++(\AndWeOffset - \i*\WireSpacing + \RegMax*\WireSpacing, 0) coordinate(and_we_\i_out_real);
                \draw (R\i_we) -- ++(0, \RegWeSpace) -| (and_we_\i_out_real);
            }

            \draw (wein) -| (weconn_0);

            % Connect the mux outputs to the Rd outputs
            \foreach \i in {1,...,2} {
                \draw (muxRd\i.rpin 1) -- ++(1.5, 0) to[font=\tiny, multiwire, l=\code{BIT\_DEPTH}] ++(0.5,0) node [anchor=west]{RD\i};
            }

        \end{circuitikz}
        \caption{Register File block diagram}
        \label{fig:block}
    \end{figure}

	Each register in figure \ref{fig:block} work with 3 input signals. When
	the \code {we} signal is \code{HIGH}, the \code{wd} bus will be written to
	the register. The \code {clk} signal will drive the D-flip-flops inside the
	register that are used as memory. This signal is falling-edge driven.
	The register output signal \code{rd} will output the contents of the register
	on a falling-edge of the clock signal.    
\\

    Figure \ref{fig:block} shows a block diagram of a register file with 8 registers. \code{Addr1} and \code{Addr2} control the data in the outputs
    \code{RD1} and \code{RD2}.
    These inputs select which register will we read from and appear in the output
    bus.
    The \code{Addr3} input will control which register to write to when \code{we}
    (write enable) is \code{HIGH}.
    The data written to the selected register will be the \code{wd}.
\\ 
 
    This register file can be implemented in VHDL by creating a table
    (array of arrays).
    The number of columns will be controlled by generic parameter
    \code{BIT\_DEPTH}.
    Each row in the table will represent a single register.
    The number of registers is controlled by the \code{LOG\_PORT\_DEPTH} which is
    number of bits to address a single register.
    Therefore the number of registers will be $2^{LOG\_PORT\_DEPTH}$.
    
    The zero register \code{R0} is designed to not be writable and always output
    \code{0} when read from.

    \section*{Results \& Analysis}
    To test the VHDL source code, a test-bench was written to analyse the 
    behaviour of the circuit and all of its operations. A table was created
    with inputs and expected outputs. The expected were checked against simulation
    outputs to verify the fidelity of the register file implementation.
    When expected results differ from simulation results, an assertion failure
    is tripped marking the simulation with failure.

    A behavioural waveform was generated.

    \begin{figure}[h!]
        \centering
        \includegraphics[width=\textwidth]{img/behaviour}
        \caption{Screen capture of register file behavioural simulation}
        \label{fig:behave}
    \end{figure}

    Figure \ref{fig:behave} shows the waveforms generated from running a behavioural simulation of the VHDL
    test-bench.

    \pagebreak

    Following the behavioural simulation, a Post-Implementation Timing simulation was performed.
    A corresponding waveform was generated.

    \begin{figure}[h!]
        \centering
        \includegraphics[width=\textwidth]{img/synthesis-timing}
        \caption{Screen capture of timing simulation in 32-bits}
        \label{fig:timing}
    \end{figure}

    The simulation test-bench is identical to the one shown in Figure \ref{fig:behave}.
    In this simulation, the gate delays associated with the Baysis 3 FPGA were taken into account.
    The output signals are seen to change values several nano-seconds after the inputs change.

    \subsection*{Testcases}
    There are two sets of testcases provided in simulations shown in Figure \ref{fig:behave} and \ref{fig:timing}.
    The first set labeled generic testcases are a set of four different inputs run on every operation.
    The ALU operations are all run in parallel and shown an different signals.
    The \code{B} input on all four of these test cases is always chosen to be low so that the shift operations
    will not overflow.
    Two of the \code{A} inputs are chosen such that the most significant bit is \code{1} so that the difference between
    \code{SRL} and \code{SRA} may be tested.
    The right shift operations can be seen to work properly as the arithmetic shift correctly fills the left bits with
    \code{1} and the logic shift fill these bits with \code{0}.
    All other operations show expected results.

    The second set of testcases labeled edge cases are a set of special input/output/operation combinations used to
    verify the fidelity of each implemented operation.
    These operations provide additional spot checking to verify the ALU implementation.

    \section*{Conclusion}
    This laboratory exercise revisited the basic of VHDL to create a six function ALU with \code{OR}, \code{XOR},
    \code{AND}, \code{SLL}, \code{SRL}, \code{SRA} operations.
    A test-bench was created to verify the output signals generated by the synthesized ALU module.
    This exercise was successful as all of the expected outputs were generated as well as reasonable timing delays
    during the timing simulation for the Baysis 3 FPGA.

    \pagebreak
    \section*{Demo results}
    \begin{figure}[h!]
        \centering
        %\includegraphics[width=\textwidth]{img/behaviour_4}
        \caption{Behavioural simulation 4-bits}
        %! suppress = FigureNotReferenced
        \label{fig:demo1}
    \end{figure}

    \begin{figure}[h!]
        \centering
        %\includegraphics[width=\textwidth]{img/post_synthesis_time_4}
        \caption{Post-synthesis timing simulation}
        %! suppress = FigureNotReferenced
        \label{fig:demo2}
    \end{figure}
    \begin{figure}[h!]
        \centering
        %\includegraphics[width=\textwidth]{img/rtl_4_schem}
        \caption{Synthesis Schematic}
        %! suppress = FigureNotReferenced
        \label{fig:demo3}
    \end{figure}
    \begin{figure}[h!]
        \centering
        %\includegraphics[width=\textwidth]{img/util_4}
        \caption{Synthesis Utilization Report}
        %! suppress = FigureNotReferenced
        \label{fig:demo4}
    \end{figure}
    \begin{figure}[h!]
        \centering
        %\includegraphics[width=\textwidth]{img/timing_4}
        \caption{Post-Implementation Timing simulation}
        %! suppress = FigureNotReferenced
        \label{fig:demo5}
    \end{figure}
    \begin{figure}[h!]
        \centering
        %\includegraphics[width=\textwidth]{img/srl_schem}
        \caption{SRL Schematic diagram}
        %! suppress = FigureNotReferenced
        \label{fig:demo6}
    \end{figure}
    \begin{figure}[h!]
        \centering
        %\includegraphics[width=\textwidth]{img/behaviour_32}
        \caption{Behavioural simulation 32-bits}
        %! suppress = FigureNotReferenced
        \label{fig:demo7}
    \end{figure}
    \begin{figure}[h!]
        \centering
        %\includegraphics[width=\textwidth]{img/timing_32}
        \caption{Post-Implementation Timing simulation 32-bits}
        %! suppress = FigureNotReferenced
        \label{fig:demo8}
    \end{figure}

\end{document}