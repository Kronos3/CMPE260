%! Author = tumbar
%! Date = 1/31/21

% Preamble
\documentclass[11pt]{article}

% Packages
\usepackage{amsmath}
\usepackage[pdfborder={0 0 0},plainpages=false,pdfpagelabels]{hyperref}
\def\code#1{\texttt{#1}}

% Document
\begin{document}
    \section*{Lab 2 prelab}
    Andrei Tumbar

    \subsection*{What is Assembly Language?}
    Assembly Language is the human readable representation of machine instructions.
    Unlike high level programming languages, assembly language only includes three main functors:
    Instructions, Registers, and Constants.

    \subsection*{In the assembly code sub a,b,c, which operand is the destination?}
    In MIPS assembly, the instruction \code{sub a,b,c} will place \code{b - c} into \code{a}.
    In every MIPS instruction, destination is first.
    
    \subsection*{Why are operands stored in registers?}
    Operands are stored in registers to allow operations to be performed quickly on arbitrary data
    instead of having to always read from memory.

    \subsection*{This lab creates 8 registers. How many registers does a MIPs have?}
    There are 32 registers in a MIPs processor.
    
    \subsection*{How many operands can be read from a MIPs register at a time?}
    A maximum of 2 registers can be read at one time
    
    \subsection*{How many operands can be written to a MIPs register at a time?}
    One register can be written to at one time



\end{document}