%! Author = tumbar
%! Date = 1/31/21

% Preamble
\documentclass[11pt]{article}

% Packages
\usepackage{amsmath}
\usepackage[pdfborder={0 0 0},plainpages=false,pdfpagelabels]{hyperref}
\usepackage{tikz}
\usepackage{circuitikz}
\usetikzlibrary{circuits.logic.IEC,calc}
\def\code#1{\texttt{#1}}

\makeatletter
\newcommand\currentcoordinate{\the\tikz@lastxsaved,\the\tikz@lastysaved}

% Document
\begin{document}
    \section*{Lab 2 prelab}
    Andrei Tumbar

    \subsection*{What is Assembly Language?}
    Assembly Language is the human readable representation of machine instructions.
    Unlike high level programming languages, assembly language only includes three main functors:
    Instructions, Registers, and Constants.

    \subsection*{In the assembly code sub a,b,c, which operand is the destination?}
    In MIPS assembly, the instruction \code{sub a,b,c} will place \code{b - c} into \code{a}.
    In every MIPS instruction, destination is first.
    
    \subsection*{Why are operands stored in registers?}
    Operands are stored in registers to allow operations to be performed quickly on arbitrary data
    instead of having to always read from memory.

    \subsection*{This lab creates 8 registers. How many registers does a MIPs have?}
    There are 32 registers in a MIPs processor.
    
    \subsection*{How many operands can be read from a MIPs register at a time?}
    A maximum of 2 registers can be read at one time
    
    \subsection*{How many operands can be written to a MIPs register at a time?}
    One register can be written to at one time

    \begin{figure}[!ht]
        \centering
        %! suppress = Ellipsis
        \begin{circuitikz}[american, circuit logic IEC]

            %\draw[step=1cm,gray,very thin] (-2,-2) grid (10,20);%just to help place nodes

            \tikzset{mux8/.style={muxdemux,muxdemux def={Lh=8, NL=8, Rh=4, NB=1, w=1, square pins=0}}}
            \tikzset{mux2/.style={muxdemux,muxdemux def={Lh=2, NL=2, Rh=2, NB=1, w=1, square pins=0}}}
            \tikzset{register/.style={qfpchip,
            hide numbers, num pins=8, external pins width=0}}

            \def\RegMax{7};
            \def\RegSpacing{1.8}
            \def\RegYStart{10}

            \def\RegPinLabels{
                1/wd/right,
                2/clk/right,
                6/rd/left,
                7/we/below};

            \foreach \i in {0,...,\RegMax} {
                \draw (5, -1 * \i * \RegSpacing + \RegYStart) node[register](R\i){R\i};

                \foreach \pin/\pinname/\labelpos in \RegPinLabels {
                    \node [\labelpos, font=\tiny] at (R\i.bpin \pin) {\pinname};
                    \draw (R\i.bpin \pin) coordinate(R\i_\pinname);
                }
            }

            % Draw the input nodes
            \draw (0,0) coordinate(addr1);
            \draw (0,-0.5) coordinate(addr2);
            \draw (0, 4) coordinate(wdin);
            \draw (1.4
            , 12) coordinate(we);
            \draw (addr1) to[font=\tiny, multiwire, l_=\code{LOG\_PORT\_DEPTH}] ++(-0.5,0) node [anchor=east]{Addr1};
            \draw (addr2) to[multiwire] ++(-0.5,0) node [anchor=east]{Addr2};
            \draw (wdin) to[font=\tiny, multiwire, l_=\code{BIT\_DEPTH}] ++(-0.5,0) node [anchor=east]{wd};
            \draw (we) node [anchor=south]{we};

            \node [mux8](muxRd1) at (9.5, 8) {};
            \node [mux8](muxRd2) at (9.5, 2) {};

            % Connect the rd of every register to the two RDout muxes
            \def\MuxOffsetA{0.1}
            \def\MuxOffsetB{0.4}
            \def\WireSpacing{0.15}
            \foreach \i [evaluate=\i as \pin using int(\i+1)] in {0,...,\RegMax} {
                \draw (R\i_rd) -- ++(\MuxOffsetA + \i*\WireSpacing, 0) |- (muxRd1.lpin \pin);
                \draw (muxRd1.lpin \pin) ++(-\MuxOffsetB - \WireSpacing*\i, 0)
                        to [short, *-] (\currentcoordinate |- muxRd2.lpin \pin) -- (muxRd2.lpin \pin);
            }

            % Connect Addr1 and Addr2 to the Mux selects
            \foreach \i in {1,...,2} {
                \draw (addr\i) -| ++(2 - \WireSpacing*\i, -4) coordinate(addr\i_inter);
            }
            \draw (addr2_inter) -| (muxRd2.bpin 1);
            \draw (addr1_inter) -- (\currentcoordinate -| muxRd2.bpin 1) -- ++(1, 0) |- (muxRd1.bpin 1);

            % Connect the wd input to every data line for registers R0-R7
            \draw (wdin) -- ++(3.8, 0) coordinate(wdconnect);
            \foreach \i in {0,...,\RegMax} {
                \draw (wdconnect) to [short, *-*] (\currentcoordinate |- R\i_wd) -- (R\i_wd);
            }

            % Draw a decoder and label the outputs
            \def\RegWeSpace{0.2}
            \def\AndWeOffset{0.2}

            \node[and gate,inputs={nnn},and gate IEC symbol={Decoder},text height=5cm,text width=2cm] (A) at (0, 9) {};
            \draw  ([xshift=-10pt]A.input 2) node[left] {Addr3} -- (A.input 2);
            \foreach \C/\B [count=\i from 0] in {0.111/000,.222/001,.333/010,.444/011,.555/100,.666/101,.777/110,.888/111} {
                \draw ( $ (A.north east)!\C!(A.south east) $ ) -- ++(10pt,0) node[left,xshift=-10] {$\B$} coordinate (decoder_\i);

                % Place a corresponding AND gate
                % Connect the inputs
                \draw (decoder_\i) node[and port,scale=0.4, anchor=in 1] (and_we_\i) {};
                \draw (and_we_\i.in 2) to [short, -*] (we |- \currentcoordinate);

                % Connect the outputs
                \draw (and_we_\i.out) -- ++(\AndWeOffset - \i*\WireSpacing + \RegMax*\WireSpacing, 0) coordinate(and_we_\i_out_real);
                \draw (R\i_we) -- ++(0, \RegWeSpace) -| (and_we_\i_out_real);
            }

            \draw (we) |- (and_we_\RegMax.in 2);

            % Connect the mux outputs to the Rd outputs
            \foreach \i in {1,...,2} {
                \draw (muxRd\i.rpin 1) -- ++(1.5, 0) to[font=\tiny, multiwire, l=\code{BIT\_DEPTH}] ++(0.5,0) node [anchor=west]{RD\i};
            }

        \end{circuitikz}
        \caption{Register File block diagram}
        \label{fig:register_file}
    \end{figure}

\end{document}