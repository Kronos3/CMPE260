\documentclass[11pt]{article}
    \title{\textbf{Homework 7}}
    \author{Andrei Tumbar}
    \date{02-24-2021}

\usepackage{circuitikzgit}
\usepackage{tikz}
\usepackage{multirow}
\usepackage{titlesec}
\usepackage{float}
\usepackage{pgfplots, pgfplotstable}
\usepackage{lmodern}
\usepackage{makecell}
\usepackage{siunitx}
\usepackage{listings}
\usepackage{subcaption}
\usepackage{kmap}
\usepackage[left=1in, right=1in, bottom=0in]{geometry}

%XeLaTeX packages
\usepackage{xltxtra}
\usepackage{fontspec} %Font package
\usepackage{xunicode}
\usepackage{inconsolata}

\setmonofont{inconsolata}

\addtolength{\topmargin}{-.875in}

\usepackage{amsmath}

\DeclareFontFamily{U}{mathx}{\hyphenchar\font45}
\DeclareFontShape{U}{mathx}{m}{n}{ <-> mathx10 }{}
\DeclareSymbolFont{mathx}{U}{mathx}{m}{n}
\DeclareFontSubstitution{U}{mathx}{m}{n}
\DeclareMathAccent{\widebar}{\mathalpha}{mathx}{"73}

\makeatletter
\newcommand{\cwidebar}[2][0]{{\mathpalette\@cwidebar{{#1}{#2}}}}
\newcommand{\@cwidebar}[2]{\@cwideb@r{#1}#2}
\newcommand{\@cwideb@r}[3]{%
  \sbox\z@{$\m@th#1\mkern-#2mu#3\mkern#2mu$}%
  \widebar{\box\z@}%
}
\newcommand\currentcoordinate{\the\tikz@lastxsaved,\the\tikz@lastysaved}
\makeatother

\usetikzlibrary{circuits.logic.IEC,calc}

\setlength{\parindent}{0pt}
\newcommand*\xor{\oplus}
\def\code#1{\texttt{#1}}
\lstset{basicstyle=\footnotesize\ttfamily,breaklines=true}

\ctikzset{logic ports=ieee}
\ctikzset{tripoles/pmos style/emptycircle}
\tikzset{dff/.style={flipflop, flipflop def={
    t1=D, t2=clk, t6=Q, nd=1}},
}
\ctikzset{
        logic ports/scale=0.7,
}

\titleformat{\section}
  {\normalfont\fontsize{12}{15}\bfseries}{\thesection}{1em}{}

\begin{document}

\maketitle
\thispagestyle{empty}

\section*{Problem 9}
Create a circuit that outputs \code{1} if four consecutive inputs are equal.

\begin{figure}[h!]
    \centering
    %! suppress = Ellipsis
    \begin{circuitikz}[american, ]

	% Third and final set
	\draw (0, 0)
			node[ieeestd xnor port, scale=0.8](xnor_1) {};
	\draw (xnor_1.in 1) -- ++(-0.2, 0)
				node[label={[font=\footnotesize]left:$w_1$}] {}
			
			(xnor_1.in 2) -- ++(-0.2, 0)
				node[label={[font=\footnotesize]left:$w_2$}] {}
			;
	

    \draw (xnor_1.out) -- ++(0.2, 0) coordinate(I1)
    		node[flipflop D, scale=0.7, anchor=pin 1](D1){};
    \draw (D1.pin 6) -- ++(0.2, 0) coordinate(I2)
    		node[flipflop D, scale=0.7, anchor=pin 1](D2){};
    \draw (D2.pin 6) -- ++(0.2, 0) coordinate(I3)
    		node[flipflop D, scale=0.7, anchor=pin 1](D3){};
    \draw (D3.pin 6) -- ++(0.2, 0) coordinate(I4);
    
    \draw (I1) ++(0.9, 1.5)
			node[ieeestd and port, scale=0.8, rotate=90](or_1) {};
    \draw (I3) ++(0.9, 1.5)
			node[ieeestd and port, scale=0.8, rotate=90](or_2) {};
    
    \draw (I1) to [short, *-] (or_1.in 1 -| \currentcoordinate) -- (or_1.in 1);
    \draw (I2) to [short, *-] (or_1.in 2 -| \currentcoordinate) -- (or_1.in 2);

    \draw (I3) to [short, *-] (or_2.in 1 -| \currentcoordinate) -- (or_2.in 1);
    \draw (I4) to [short, -] (or_2.in 2 -| \currentcoordinate) -- (or_2.in 2);
    
    \draw (or_1.out)
    		node[ieeestd and port, scale=0.8, rotate=90, anchor=in 1](out_and){};
    \draw (or_2.out) -| (out_and.in 2);
    
    \draw (out_and.out)
				node[label={[font=\footnotesize]above:$z$}] {};

    \end{circuitikz}
    \caption{Circuit which compares two numbers and outputs a \code{1} if they are equal for four consecutive clocks.}
    \label{fig:block}
\end{figure}

The comparison is done via a \code{XNOR} gate which will output a \code{1} if the 
two inputs are equal. The previous 3 comparisons will be held in the D-flip-flop.
All of the previous comparisons as well as the current comparison must be true
in order to produce a high output.

\section*{Problem 10}
(See VHDL file)

\pagebreak
\section*{Exercise 22}
This FSM will output a \code{1} if the sequence \code{AB} is seen in the input.
It will always output \code{0} otherwise.


\begin{table}[H]
    \renewcommand{\arraystretch}{1.2}
    % \setlength{\tabcolsep}{12pt}
    \caption{State-transistion and output table}
    \begin{center}
        \begin{tabular}{|c|c|c|c|}
            \hline
            Input & State & Next State & Output\\\hline
            \hline
            $A$ & \code{00} & \code{01} & \code{0}\\\hline
            $\widebar{A}$ & \code{00} & \code{00} & \code{0}\\\Xhline{4\arrayrulewidth}
            
            $B$ & \code{01} & \code{10} & \code{0}\\\hline
            $\widebar{B}$ & \code{01} & \code{00} & \code{0}\\\Xhline{4\arrayrulewidth}
            
             & \code{10} & \code{00} & \code{1}\\\hline
        \end{tabular}
    \end{center}
    \label{tab:decode}
\end{table}

\begin{figure}[h!]
	\begin{subfigure}{.5\textwidth}
		\centering
		\begin{Karnaugh}{$Q_1$$Q_0$}{AB}
				\contingut{0,0,0,0,
						   0,1,0,0,
						   0,0,0,0,
						   0,1,0,0}
		   \implicant{5}{13}{yellow}
		\end{Karnaugh}
		\caption{K-map for ${Q_1}^{\prime}$}
		\label{fig:sop}
	\end{subfigure}
	\begin{subfigure}{.5\textwidth}
		\centering
		\begin{Karnaugh}{$Q_1$$Q_0$}{AB}
				\contingut{0,0,0,0,
						   0,0,0,0,
						   1,0,0,0,
						   1,0,0,0}
		   \implicant{12}{8}{red}
		\end{Karnaugh}
		\caption{K-map for ${Q_0}^{\prime}$}
		\label{fig:pos}
	\end{subfigure}
	
	
	
	\caption{K-Map for Exercise-22 state machine}
	\label{fig:kmap}
\end{figure}

\begin{align*}
Y &= Q_1\widebar{Q_0}\\
{Q_1}^{\prime} &= \widebar{Q_1}Q_0B\\
{Q_0}^{\prime} &= \widebar{Q_0}\widebar{Q_0}A
\end{align*}

\end{document}

