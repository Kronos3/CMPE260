% Author: Andrei Tumbar

\documentclass[11pt]{article}
    \title{\textbf{Homework 11}}
    \author{Andrei Tumbar}
    \date{04-14-2021}

\usepackage{pgf}
\usepackage{tikz}
\usepackage{amsmath}
\usepackage{float}
\usetikzlibrary{arrows,automata}
\addtolength{\topmargin}{-1.2in}
\begin{document}
\maketitle

\section*{Exercise 34}
\subsubsection*{a}

The critical path on this circuit is from $A$ or $B$ on the first adder
to $C_{out}$ followed by $C_{in}$ to $S$ on the second adder. The propagation
delay of the flip-flops as well as their hold times are also taken into account.

\begin{align*}
T_{c} &= 35 + 30 + 25 + 20\\
&= 35ps + 25ps + 20ps + 10ps = 110ps\\
F &= 9.09 GHz
\end{align*}

\subsubsection*{b}

\begin{align*}
T_{c} &= 35 + 30 + 25 + 20 + t_{skew}\\
1/8 GHz &= 125ps\\
t_{skew} &= 125ps - 110ps = 15ps
\end{align*}

\subsubsection*{b}
The flip flops have a minimum hold time of $10$ps. The earliest the output
can begin to change is the sum of the $FF_{cd}$ and the $Adder_{min-cd}$
which is:

\begin{align*}
CD_{min} &= 21ps + 15ps = 35ps
\end{align*}

Therefore the max skew for a holdtime violation is:

\begin{align*}
Skew_{max} &= 35ps - 10ps = 25ps
\end{align*}

\section*{Exercise 35}
\subsubsection*{a}

\begin{align*}
40 MHz &= 25ns\\
25ns &= 0.61x + 0.72ns + 0.53ns\\
x &= 38.93 = 38 CLBs
\end{align*}
\subsubsection*{b}
The hold time is 0ns so the maximum clock skew is the clock period of 25ns.

\section*{Exercise 40}
Given that the metastablity detector M has a low enough propagation time, this
circuit would never produce a metastable result. This setup time would need to be
low enough to satify this equation:

\begin{align*}
T_c \geq FF_{setup} + 2*FF_{prop} + M_{prop}
\end{align*}

The FF propagation time must be multiplied by 2 so that the D2 signal has time to
reset when the metastable detector asynchronously resets the first flip-flop.

\end{document}
